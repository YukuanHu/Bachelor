\documentclass[10pt]{beamer}

%\logo{\includegraphics[width=.1\textwidth]{logo.jpg}}
\setbeamertemplate{background}{\includegraphics[height=\paperheight]{logo.jpg}}
\ifx\pdfoutput\undefined
% we are running LaTeX, not pdflatex
\usepackage{graphicx}
\else
% we are running pdflatex, so convert .eps files to .pdf
\usepackage{graphicx}
\usepackage{epstopdf}
\fi
\usepackage{xeCJK}
%\usepackage{fontspec,xunicode,xltxtra}
\setCJKmainfont[BoldFont=SimHei]{KaiTi}
\setCJKmonofont{SimSun}     % 设置缺省中文字体
%\setmainfont[BoldFont=SimHei]{KaiTi}
%\setmonofont{SimSun}     % 设置缺省中文字体
%\usepackage[noindent]{ctex}
\usepackage{amsmath}
\usepackage{multicol}
\usepackage{fontspec}
\usepackage{tikz}  


\setCJKfamilyfont{song}{SimSun}
\newcommand*{\songti}{\CJKfamily{song}}

%\renewcommand{\figurename}{图}

%\input{beihangbeamerstyle/beihangcolor}
%\input{beihangbeamerstyle/beihangbeamerstyle}

\title{\Large 上海市优秀毕业生答辩}
\date{致远楼108, 2019.5.13}
\author{胡雨宽$\quad$1553396\\
\small 同济大学 数学科学学院\\
\small 2015级理科班$\:\:$数学与应用数学}

\AtBeginSection[]
{
  \begin{frame}<beamer>
    \frametitle{提纲}
    \tableofcontents[sections={2-},currentsection,subsectionstyle=show/show/hide]%,sections={2-6}]
  \end{frame}
  \addtocounter{framenumber}{-1}
}

\AtBeginSubsection[]
{
  \begin{frame}<beamer>
    \frametitle{提纲}
    \tableofcontents[sections={2-},currentsection,subsectionstyle=show/shaded/hide]
  \end{frame}
  \addtocounter{framenumber}{-1}
}

\begin{document}
%----------------------------------------------------------------------
% Title frame
\begin{frame}[plain]
\maketitle
\end{frame}

%----------------------------------------------------------------------
% Outline frame
% PLEASE RUN pdflatex TWICE 
\section*{提纲}
\begin{frame}
\frametitle{提纲}
\tableofcontents[sections={2-},hideallsubsections]
\end{frame}
%%=====================================================================
% Section I
%\section{个人简介}
%----------------------------------------------------------------------
% Content frame
%\begin{frame}
%\frametitle{个人简介}
%\begin{columns}
%			\begin{column}{0.4\textwidth}
%				\centering
%				\includegraphics[height = .7\textwidth]{photo.jpg}
%			\end{column}
%			\begin{column}{.6\textwidth}
%				\begin{itemize}
%					\item 姓名: 胡雨宽
%					\item 性别: 男
%					\item 班级: 2015级理科班
%					\item 籍贯: 江西南昌
%				\end{itemize}
%			\end{column}
%\end{columns}
%\end{frame}

%%=====================================================================
% Section II
\section{学习方面}
%----------------------------------------------------------------------
\begin{frame}
\frametitle{学习方面——锐意进取$\,$开拓创新}
%\begin{columns}
%\begin{column}{0.3\textwidth}
%\only<2>{
%\begin{center}
%\includegraphics[width=.2\paperwidth]{Silkroad.jpg}\\
%\footnotesize \textbf{图}1: 丝路数学中心第3次学术活动\end{center}}
%\only<3>{\begin{center}\includegraphics[width=.23\paperwidth]{Silkroadphoto.jpg}\\
%\footnotesize \textbf{图}2: 与香港中文大学(深圳)副校长罗智泉教授的合影\end{center}}
%\only<4>{
%\begin{center}
%\begin{tikzpicture}
%  \node (img1) {\includegraphics[height=3cm]{15-16一等奖.jpg}};
%  \node (img2) at (img1.south east) {\includegraphics[height=3cm]{16-17一等奖.jpg}};
%\end{tikzpicture}
%\footnotesize \textbf{图}3: 奖学金证书\end{center}}
%\only<5>{
%\begin{center}
%\includegraphics[height=2cm]{15-16优秀学生.jpg}\\
%\includegraphics[height=2cm]{16-17优秀学生标兵.jpg}\\
%\footnotesize \textbf{图}4: 优秀学生证书
%\end{center}
%}
%\only<6>{
%\begin{center}
%\begin{tikzpicture}
%  \node (img1) {\includegraphics[height=3cm]{CET4.jpg}};
%  \node (img2) at (img1.south east) {\includegraphics[height=3cm]{CET6.jpg}};
%\end{tikzpicture}
%\footnotesize \textbf{图}5: 外语成绩证明\end{center}
%}
%\end{column}
%\begin{column}{0.8\textwidth}
\begin{itemize}
\item 学习成绩: 7学期平均绩点4.85, {\color{red}{专业第二}}.\\
$\quad\quad\quad\quad\,$专业课绩点4.92.
\item 保研情况: 现已{\color{red}{保研至中国科学院数学与系统科学\\
$\quad\quad\quad\quad\,\,$研究院 (AMSS)}}, 方向为最优化计算.\\
$\quad\quad\quad\quad\,\,$硕博连读. {\color{red}{访问学者}} (2019.3-5).
\item 奖学金情况: {\color{red}连续3年获得本科生一等奖学金}.\\
$\quad\quad\quad\quad\quad\:\:$社会奖学金 (2017-2018).
\item 荣誉称号: ``优秀学生''称号 (2015-2016, 2017-2018).\\
$\quad\quad\quad\quad\:\:\:${\color{red}{``优秀学生标兵''称号 (2016-2017)}}.
\item 外语考试: CET4 655分,$\,${\color{red}{CET6 610分}}
\end{itemize}

%\end{column}
%\end{columns}
\end{frame}

\begin{frame}
\frametitle{学习方面——锐意进取$\,$开拓创新 (续)}
%\begin{columns}
%\begin{column}{0.3\textwidth}
%\only<2>{
%\begin{center}
%\begin{tikzpicture}
%  \node (img1) {\includegraphics[height=2cm]{2016校赛建模三等奖.jpg}};
%  \node (img2) at (img1.south) {\includegraphics[height=2cm]{2017校赛建模三等奖.jpg}};
%  \node (img3) at (img2.south) {\includegraphics[height=2cm]{2018校赛建模三等奖.jpg}};
%\end{tikzpicture}
%\footnotesize \textbf{图}6: 数学建模竞赛 \\(校赛)\end{center}
%}
%\only<3>{
%\begin{center}
%\begin{tikzpicture}
%  \node (img1) {\includegraphics[height=2cm]{2017国赛建模三等奖.jpg}};
%  \node (img2) at (img1.south east) {\includegraphics[height=2.5cm]{2018美赛二等奖.jpg}};
%\end{tikzpicture}
%\footnotesize \textbf{图}7: 数学建模竞赛\\ (国赛+美赛)
%\end{center}
%}
%\only<4>{
%\begin{center}
%\includegraphics[width=3cm]{2017数学竞赛二等奖.jpg}\\
%\footnotesize \textbf{图}8: 数学竞赛
%\end{center}
%}
%\only<5>{
%\begin{center}
%\includegraphics[width=3cm]{NIC.jpg}\\
%\footnotesize \textbf{图}9: 国创结题证书
%\end{center}
%}
%\end{column}
%\begin{column}{0.8\textwidth}
{\color{blue}{\textbf{科创竞赛}}}
\begin{itemize}
\item 数学建模: 校赛三等奖 (2016-2018, 团队, {\color{red}{队长}}).\\
$\quad\quad\quad\quad\:\:$国赛上海市三等奖 (2017, 团队, {\color{red}{队长}}).\\
$\quad\quad\quad\quad\:\:$美赛二等奖 (2018, 团队, {\color{red}{队长}}).
\item 数学竞赛: 校赛 (数学类)二等奖 (2017).
\item 创新项目: 机器学习在多音源环境语音识别中的应用\\
$\quad\quad\quad\quad\:\:\,\,$(2017.4-2019.4, {\color{red}{国创}}, {\color{red}{负责人}}, 已结题.)
\end{itemize}
%\end{column}
%\end{columns}
\end{frame}

\begin{frame}
\frametitle{学习方面——锐意进取$\,$开拓创新 (续)}
{\color{blue}{\textbf{科创竞赛}}}:\\
{\color{red}{2018年``浦发信用卡杯''中国高校SAS数据分析大赛全国总冠军, 我校历年\textbf{最好成绩} (团队, 队长).}}
\begin{figure}\centering
\includegraphics[width=.35\paperwidth]{SAStogether.jpg}$\quad$
\includegraphics[width=.215\paperwidth]{SASphoto.jpg}\\\vspace{1em}
\includegraphics[width=.25\paperwidth]{SAScertificate.jpg}
\end{figure}
\begin{center}\footnotesize \textbf{图}1: 2018.11北京钓鱼台国宾馆颁奖现场\end{center}
\end{frame}

\section{实践与活动方面}
\begin{frame}
\frametitle{实践与活动方面——以身作则$\,$热心服务}
\onslide<1->{{\color{blue}{\textbf{学院层面}}}:}
\begin{itemize}
\item \onslide<2->{班级: 担任班长 (2018.4-至今).}
\item \onslide<2->{学生会: 学术部干事(2015.9-2017.9), ``光阴书会''.}
\item \onslide<2->{学院活动: 话剧比赛 (2016, 2018).\\
$\quad\quad\quad\quad\:\:$红歌会 (2017). \\
$\quad\quad\quad\quad\:\:$晚会 (2015, 2018).}
\item \onslide<2->{{\color{red}{当选同济大学第四十二次学代会的学生代表和数学科学学院常代表 (2018)}}.}
\end{itemize}
\end{frame}

\begin{frame}
\frametitle{实践与活动方面——以身作则$\,$热心服务 (续)}
\onslide<1->{{\color{blue}{\textbf{学院层面}}}:}\\
\onslide<1->{学生组织: 加入{\color{red}{``数学外卖''活动}}, 进入线性代数组 (2017.6).\\
$\quad\quad\quad\quad\:\:$担任线性代数组{\color{red}{组长}} (2018.4-至今).\\
\onslide<2->{$\quad\quad\:\:$1. 制作参考课件, 统一课件样式;\\
$\quad\quad\:\:$2. 充分利用组内资源, 鼓励新人上场, 任期内3场讲座、1次辅导\\
$\quad\quad\:\:$3. 完善相关制度.}\\[1em]
个人累计3场讲座、3次辅导. {\color{red}{``金牌讲师''}}. \\
{\color{red}{``2017-2018年优秀志愿服务项目''}}.}
\begin{center}
\includegraphics[width=.3\paperwidth]{linearalgebra.jpg}$\quad$
\includegraphics[width=.15\paperwidth]{gold.jpg}$\quad$
\includegraphics[width=.28\paperwidth]{slide.png}\\
\footnotesize \textbf{图}2: 左: ``全''家福; 中: ``金牌讲师''; 右: 课件一页
\end{center}
\end{frame}

\begin{frame}
\frametitle{实践与活动方面——以身作则$\,$热心服务 (续)}
\onslide<1->{{\color{blue}{\textbf{学校层面}}}:}\\
\onslide<1->{志愿服务: 义务献血(2015, 2016, 2017, 2018).} 
\only<1>{\\$\quad\quad\quad\quad\:\:\,\,$110周年校庆志愿者活动 (2017).}
\only<2>{{\\\color{red}{$\quad\quad\quad\quad\:\:\,\,$110周年校庆志愿者活动 (2017).}}}
\only<3->{\\$\quad\quad\quad\quad\:\:\,\,$110周年校庆志愿者活动 (2017).}
\only<1-2>{\\$\quad\quad\quad\quad\:\:$迎新生``小红帽''志愿者活动 (2018).}
\only<3->{{\\\color{red}{$\quad\quad\quad\quad\:\:$迎新生``小红帽''志愿者活动 (2018).}}}
\only<2>{\begin{center}
\includegraphics[width=.6\paperwidth]{520.jpg}\\
\footnotesize\textbf{图}3: 2017.5.20 我院1985级校友合影
\end{center}}
\onslide<3->{\begin{center}
\includegraphics[width=.415\paperwidth]{redhat1.jpg}$\:$
\includegraphics[width=.415\paperwidth]{redhat2.jpg}\\
\footnotesize \textbf{图}4: 2018.9 ``小红帽''活动\end{center}}
\end{frame}

\section{其他方面}
\begin{frame}
\frametitle{其他方面——全面发展$\,$持之以恒}
{\color{blue}{\textbf{体育锻炼}}}: \pause 长跑 (3km$\to$5km$\to$8km$\to$10km$\to$20km), 定协\pause
\begin{itemize}
\onslide<3->{\item 2018北京马拉松线上赛 (10km, {\color{red}{完赛}}成绩50分58秒, 第411名)}
\onslide<4->{\item 2018广州马拉松线上赛 (10km, {\color{red}{完赛}}成绩44分59秒)}
\onslide<5->{\item 2019扬州半程马拉松线上赛 (半程21.0975km, {\color{red}{完赛}}成绩1小时58分02秒)}
\end{itemize}

\begin{center}\onslide<3->{\includegraphics[width=.16\paperwidth]{bjmarathon.jpg}}$\quad$
\onslide<4->{\includegraphics[width=.165\paperwidth]{gzmarathon.jpg}}$\quad$
\onslide<5->{\includegraphics[width=.21\paperwidth]{yzmarathon.jpg}}\\
\footnotesize \onslide<5->{\textbf{图}5: 马拉松线上赛成绩截图}\end{center}
\end{frame}

\begin{frame}
\frametitle{其他方面——全面发展$\,$持之以恒 (续)}
{\color{blue}{\textbf{学校比赛}}}
\begin{itemize}
\item 健美操比赛团体一等奖 (2015, 2016).
\item 同济大学第二届``体质达人''比赛``肺活量''和``身体成分''组双料``体质达人'' (2017).
\end{itemize}
\end{frame}

\begin{frame}{致谢}
\begin{center}
\huge \songti 感谢聆听!\\[0.3em]
\normalsize
\url{huyukuan2015@tongji.edu.cn}
\end{center}
\end{frame}

\end{document}


