\documentclass{beamer}

%\setbeamertemplate{background}{\includegraphics[height=\paperheight]{logo.jpg}}
\usetheme{Warsaw}
\useoutertheme{infolines}
%\useinnertheme{rounded}
%\usecolortheme{dolphin}
%\setbeamertemplate{background}{\includegraphics[height=\paperheight]{logo.jpg}}
%\ifx\pdfoutput\undefined
% we are running LaTeX, not pdflatex
%\usepackage{graphicx}
\usepackage{float}
%\else
% we are running pdflatex, so convert .eps files to .pdf
%\usepackage{graphicx}
\usepackage{epstopdf}
%\fi
\usepackage{color}
\usepackage{subfigure}
\usepackage{xeCJK}
\usepackage{multirow}
\usepackage{enumerate}
\usepackage{fontspec}
%\usepackage{xcolor} 
\newcommand{\tnewroman}{\fontspec{Times New Roman}}
%\newcommand{\sserif}{\fontspec{sans serif}}
%\setmainfont{Times New Roman}
%\usepackage{mathspec}
%\setallmathfont{Times New Roman}
%\usepackage{fontspec,xunicode,xltxtra}
\setCJKmainfont[BoldFont=SimHei]{KaiTi}
\setCJKmonofont{SimSun}     % 设置缺省中文字体
%\setmainfont{Sans Serif}
%\setmainfont[BoldFont=SimHei]{KaiTi}
%\setmonofont{SimSun}     % 设置缺省中文字体
%\usepackage[noindent]{ctex}
\usepackage{amsmath}
\usepackage{multicol}
%\usepackage{fontspec}
\usepackage{tikz}  
\usepackage{graphicx}
\usepackage{pgf}
\usepackage{algorithm}
\usepackage{algorithmic}

\usefonttheme[onlymath]{serif}
\setCJKfamilyfont{song}{SimSun}
\newcommand*{\songti}{\CJKfamily{song}}
\newcommand*{\red}[1]{\textcolor[rgb]{0.8,0.05,0.05}{#1}}
\newcommand*{\sfred}[1]{\red{\textsf{#1}}}
\newcommand*{\beq}[1]{{\textcolor[rgb]{0,0,0.67}{\eqref{#1}}}}
\newcommand*{\blue}[1]{{\textcolor[rgb]{0,0,0.67}{#1}}}

%\setbeamertemplate{headline}{	% navigation bar of section (second headline)
%\begin{beamercolorbox}{section in head/foot}
%   \vskip2pt\insertnavigation{\paperwidth}\vskip2pt
%\end{beamercolorbox}
%\begin{beamercolorbox}[colsep=1.5pt,ht=.3ex]{upper separation line head}		% separator
%\end{beamercolorbox}
%}
\setbeamertemplate{footline}{
\leavevmode%
\hbox{%
\begin{beamercolorbox}[wd=.333333\paperwidth,ht=2.25ex,dp=1ex,center]{author in head/foot}%
    \usebeamerfont{title in head/foot}胡雨宽 (数学科学学院)
\end{beamercolorbox}%
\begin{beamercolorbox}[wd=.333333\paperwidth,ht=2.25ex,dp=1ex,center]{title in head/foot}%
    \usebeamerfont{title in head/foot} 求解一类双线性规划问题的数值算法
\end{beamercolorbox}%
\begin{beamercolorbox}[wd=.333333\paperwidth,ht=2.25ex,dp=1ex,right]{date in head/foot}%
    \usebeamerfont{date in head/foot} 2019.6.4 \hspace*{2em}
    \insertframenumber{} / \inserttotalframenumber\hspace*{2ex} 
\end{beamercolorbox}}%
\vskip0pt%
}

\setbeamertemplate{navigation symbols}{}
\setbeamercolor{upper separation line head}{bg=white}

\newcommand{\primal}{\mathrm{primal}}
\newcommand{\dual}{\mathrm{dual}}
\newcommand{\trace}{\mathrm{tr}}
\newcommand{\vectorize}{\mathrm{vec}}
\newcommand{\prox}{\mathrm{prox}}
\newcommand{\mcL}{\mathcal{L}}
\newcommand{\st}{\mathrm{s.t.}}
\newcommand{\dom}{\mathrm{dom}}
\newcommand{\diag}{\mathrm{diag}}
\newcommand{\one}{\mathbf{1}}
\AtBeginSection[]
{
  \begin{frame}<beamer>
    \frametitle{提纲}
    \tableofcontents[sections={1-},currentsection,subsectionstyle=hide/hide/hide]%,sections={2-6}]
  \end{frame}
  \addtocounter{framenumber}{-1}
}

%\AtBeginSubsection[]
%{
%  \begin{frame}<beamer>
%    \frametitle{提纲}
%    \tableofcontents[sections={2-},currentsection,subsectionstyle=show/shaded/hide]
%  \end{frame}
%  \addtocounter{framenumber}{-1}
%}
\renewcommand{\algorithmicrequire}{\textbf{输入: }}
\renewcommand{\algorithmicensure}{\textbf{输出: }}
\begin{document}
\section*{本科毕业论文答辩}
\begin{frame}
\logo{TUlogo.jpg}
\title{求解一类双线性规划问题的数值算法}
\date{\tnewroman{2019.6.4}}
\author{\quad
\begin{minipage}{4cm}
    答辩人: 胡雨宽\\
    导\quad 师: 殷俊锋\quad 教授
\end{minipage}}
\institute{同济大学数学科学学院}
\titlegraphic{\includegraphics[width=2cm]{TUlogo.jpg}}
\maketitle
\end{frame}

%\logo{\pgfputat{\pgfxy(9,0)}{\pgfbox[center,base]{\includegraphics[height=1.2cm]{TUlogo.jpg}}}}
\section*{提纲}
\begin{frame}
\frametitle{提纲}
\tableofcontents[sections={2-},hideallsubsections]
\end{frame}

\section{引言}
\subsection{问题背景与陈述}
\begin{frame}{问题背景}
\textbf{最优运输问题} (\sfred{Villani '08}): 
\begin{itemize}
\item 建立有效比较概率分布的几何工具.
\item 极小化将一概率分布"运输"到另一概率分布的"花费".
\end{itemize}

	$$\mathrm{(LP)}\quad\begin{array}{rl}
		\min\limits_{a_{ij}} & \sum\limits_{i,j}c_{ij}a_{ij}\\
		\st & \sum\limits_ja_{ij}=f_i,\quad i=1,\ldots,m,\\
		 & \sum\limits_ia_{ij}=g_j,\quad j=1,\ldots,n,\\
		 & a_{ij}\ge0,\quad i=1,\ldots,m,j=1,\ldots,n,
	\end{array}$$
	这里$c_{ij},f_i,g_j\ge0,\forall i,j$.
\end{frame}


\begin{frame}{问题陈述}
	\begin{equation}
		\begin{array}{rl}
			\min\limits_{X,Y} & \sum\limits_{i\ne j}\frac{x_{ij}}{|r_i-r_j|}+\sum\limits_{i\ne k}\frac{y_{ik}}{|r_i-r_k|}+\sum\limits_{i,j,k:j\ne k}\frac{x_{ij}y_{ik}}{|r_j-r_k|}\\
			\st & \sum\limits_jx_{ij}=\rho_i,\quad i=1,\ldots,n,\\
			 & \sum\limits_ix_{ij}=\rho_j,\quad j=1,\ldots,n,\\
			 & \sum\limits_ky_{ik}=\rho_i,\quad i=1,\ldots,n,\\
			 & \sum\limits_iy_{ik}=\rho_k,\quad i=1,\ldots,n,\\
			 & x_{ij},y_{ik}\ge0,\quad i,j,k=1,\ldots,n,\\
			 & x_{ii},y_{ii}=0,\quad i=1,\ldots,n,
		\end{array}
		\label{original problem element form}
	\end{equation}
	其中$X,Y\in\mathbb{R}^{n\times n}$, $r=(r_1,\ldots,r_n)^T,\rho=(\rho_1,\ldots,\rho_n)^T\in\mathbb{R}_+^n$. 将$\{|r_i-r_j|\}$储存于$R=(r_{ij})\in\mathbb{R}^{n\times n}$, 其中
	$$r_{ij}=\left\{\begin{array}{ll}
		1/|r_i-r_j|, & i\ne j,\\
		0, & i=j.
	\end{array}\right.$$
\end{frame}

\subsection{研究现状}
\begin{frame}{研究现状}
\begin{enumerate}[角度1]
\item \textbf{可分离约束的双线性规划.} 广泛的应用 (\sfred{Konno '71}等).\\[1em]
已有方法:
\begin{itemize}
\item 割平面法 (\sfred{Ritter '66}等).
\item 分支定界法 (\sfred{Falk '73}等).
\end{itemize}
\item \textbf{非凸二次规划.}\\[1em]
已有方法 (\sfred{Nocedal/Wright '06}):
\begin{itemize}
\item 逐步二次规划 (\tnewroman{SQP}).
\item 积极集法 (\tnewroman{Active-set Methods}).
\item 内点法 (\tnewroman{Interior-point Methods}).
\item $\ldots$
\end{itemize}\pause
%\begin{itemize}
%\item 可分离约束的双线性规划.
%\begin{itemize}
%\item 割平面法.
%\item 分支定界法.
%\item 线性化方法和对偶技术.
%\end{itemize}
%\item 非凸二次规划.
%\begin{itemize}
%\item 逐步二次规划 (\tnewroman{SQP}).
%\item 积极集法.
%\end{itemize}
%\end{itemize}
\end{enumerate}
以上方法均以列向量为求解对象$\Rightarrow$\textbf{\color{red}{计算量问题}}.
\end{frame}

\subsection{预备知识}
\begin{frame}{预备知识}
\begin{block}{标准矩阵内积}
对$\forall A,B\in\mathbb{R}^{n\times n}$, 定义
$$\langle A,B\rangle:=\trace(A^TB)=\sum\limits_{i,j}a_{ij}b_{ij}.$$
\end{block}\pause
问题\beq{original problem element form}即可化为矩阵形式:
\begin{equation}
	\begin{array}{rl}
		\min\limits_{X,Y} & \langle R,X\rangle+\langle R,Y\rangle+\langle Y,XR\rangle\\
		\st & X\one=\rho,X^T\one=\rho,\trace(X)=0,X\ge0,\\
		 & Y\one=\rho,Y^T\one=\rho,\trace(Y)=0,Y\ge0,
	\end{array}
	\label{original problem matrix form 1}
\end{equation}
其中$\one$为全\tnewroman{1}向量.
\end{frame}

\begin{frame}{预备知识 (续)}
实验表明$\ldots$
\begin{block}{假设}
问题\beq{original problem matrix form 1}的所有稳定点$(X,Y)$均满足$X=Y$.
\end{block}\pause
$\Rightarrow$简化问题:
\begin{equation}
	\begin{array}{rl}
		\min\limits_X & 2\langle X,R\rangle+\langle X,XR\rangle\\
		\st & X\one=\rho,X^T\one=\rho,\trace(X)=0,X\ge0.
	\end{array}
	\label{original problem matrix form 2}
\end{equation}\pause
引入\textbf{分裂变量}$Z\in\mathbb{R}^{n\times n}$, 进一步得到等价的
\begin{equation}
	\begin{array}{rl}
		\min\limits_{X,Z} & f(X,Z)\triangleq2\langle X,R\rangle+\langle Z,XR\rangle\\
		\st & X\one=\rho,\trace(X)=0,\\
		 & Z^T\one=\rho,Z\ge0,\\
		 & X=Z.
	\end{array}
	\label{original problem matrix form}
\end{equation}
之后\textbf{重点求解}问题\beq{original problem matrix form}.
\end{frame}

\section{最优性条件}
\begin{frame}{最优性条件}
\begin{block}{问题\eqref{original problem matrix form}的\tnewroman{KKT}条件}
若$(X^*,Z^*)$为问题\beq{original problem matrix form}的解, 则存在拉格朗日乘子$\mu^*\in\mathbb{R},\lambda_1^*,\lambda_2^*\in\mathbb{R}^n$, $\Phi^*,0\le\Omega^*\in\mathbb{R}^{n\times n}$, 使得
\begin{equation}
\left\{\begin{array}{ll}
\left.\begin{array}{l}2R+Z^*R-\lambda_1^*\one^T-\Phi^*-\mu^* I=0,\\X^*R-\one\left(\lambda_2^*\right)^T+\Phi^*-\Omega^*=0,\end{array}\right\} & \begin{array}{l}
\makebox{稳定性条件}\\\makebox{或对偶可行性条件}
\end{array}\\
\left.\begin{array}{l}
X^*\one=\rho,\trace(X^*)=0,\\
\left(Z^*\right)^T\one=\rho,Z^*\ge0,\\
\Omega^*\ge0,
\end{array}\right\} & \makebox{原始可行性条件},\\
\Omega^*\circ Z^*=0.& \makebox{互补松弛条件}.
\end{array}\right.
\label{KKT}
\end{equation}
这里``$\circ$''表示\tnewroman{Hadamard}积.
\end{block}
\end{frame}

\section{算法设计}
\subsection{\tnewroman{ADMM}算法简介}
\begin{frame}{\tnewroman{ADMM}算法简介}
\tnewroman{ADMM}算法 (介绍可见\sfred{Boyd, et al. '10})求解问题\textbf{一般形式}:
\begin{equation}
	\begin{array}{rl}
		\min\limits_{x\in\mathbb{R}^d} & \theta(x):=\sum\limits_{i=1}^n\theta_i(x_i)+\ell(x_1,\ldots,x_n)\\
		\st & \sum\limits_{i=1}^nA_ix_i=b.
	\end{array}
	\label{general ADMM problem}
\end{equation}
	现有工作简介:
	\begin{itemize}
	\item 凸可分问题 (\sfred{Boyd, et al. '10}, \sfred{He/Yuan '12}等): $n\ge2,\ell=0$, $\theta_1,\theta_2$是凸函数. 可能发散 (\sfred{Chen, et al. '16}).$\Rightarrow$\textbf{无假设的困境}.\pause
	\item 凸不可分问题 (\sfred{Hong, et al. '14}, \sfred{Chen, et al. '19}). 即使$n=2,\theta(\cdot)$凸, 仍然开放 (\sfred{Hong/Luo/Razaviyayn '16}).
	\item 非凸问题. 理论缺乏, 应用广泛. 
	\begin{itemize}
	\item \sfred{Wen, et al. '13}等: 需要强加无法验证的条件.
	\end{itemize}
	\end{itemize}
\end{frame}

\begin{frame}{\tnewroman{ADMM}算法简介 (续)}
\begin{equation}\mcL_A(X,Z,\Phi)=f(X,Z)-\langle\Phi,X-Z\rangle+\frac{\beta}{2}\Vert X-Z\Vert_F^2.\end{equation}
\begin{algorithm}[H]
\floatname{algorithm}{框架}
\caption{求解问题(4)的\tnewroman{ADMM}算法框架}
\begin{algorithmic}[1]
\REQUIRE $X^0,Z^0,\Phi^0,\beta^0,k:=0$.
\ENSURE $X^k,Z^k,\Phi^k$.
\WHILE {收敛性测试未通过}
\STATE {$X^{k+1}=\arg\min\limits_{X:X\one=\rho,\trace(X)=0}\mcL_A(X,Z^k,\Phi^k)$;}
\STATE {$Z^{k+1}=\arg\min\limits_{Z:Z^T\one=\rho,Z\ge0}\mcL_A(X^{k+1},Z,\Phi^k)$;}
\STATE {更新$\Phi^k$得到$\Phi^{k+1}$;}
\STATE {如有需要, 更新$\beta^k$得到$\beta^{k+1}$;}
\STATE {$k:=k+1$;}
\ENDWHILE
\end{algorithmic}
\end{algorithm}
\end{frame}

\subsection{子问题的求解}
\begin{frame}{子问题的求解--$X$子问题}
省去上标$k$, 改用`$+$'标记更新值. 
\begin{equation}
	\begin{array}{rl}
		\min\limits_X & 2\langle X,R\rangle+\langle Z,XR\rangle-\langle\Phi,X-Z\rangle+\frac{\beta}{2}\Vert X-Z\Vert_F^2\\
		\st & X\one=\rho,\quad\trace(X)=0.
	\end{array}
	\label{X subproblem}
\end{equation}\pause
$$\begin{aligned}
	M_1&=2R\one+ZR\one-\Phi\one-\beta Z\one+\beta\rho,\\
	m_1&=2\trace(R)+\trace(ZR)-\trace(\Phi)-\beta\trace(Z).
\end{aligned}$$
\begin{equation}
	\mu=\frac{1}{n-1}\left(-\frac{1}{n}\one^TM_1+m_1\right),\quad\lambda_1=\frac{1}{n}(M_1-\one\mu).\label{X multiplier}
\end{equation}
\begin{equation}
	X^+=-\frac{1}{\beta}(2R+ZR-\lambda_1\one^T-\mu I-\Phi-\beta Z).\label{pre solve X}
\end{equation}
\beq{X multiplier}和\beq{pre solve X}给出$X$子问题\beq{X subproblem}的解$X^+$.
\end{frame}

\begin{frame}{子问题的求解--$Z$子问题}
\begin{equation}
	\begin{array}{rl}
		\min\limits_Z & \langle Z,X^+R\rangle-\langle\Phi,X^+-Z\rangle+\frac{\beta}{2}\Vert X^+-Z\Vert_F^2\\
		\st & Z^T\one=\rho,\quad Z\ge0.
	\end{array}
	\label{Z subproblem}
\end{equation}
\pause
忽略非负约束, 求超平面$Z^T\one=\rho$上的一点$\widetilde Z$:
$$\lambda_2=\frac{1}{n}\left[R\left(X^+\right)^T\one+\Phi^T\one-\beta\left(X^+\right)^T\one+\beta\rho\right],$$
$$\widetilde Z=\frac{1}{\beta}(X^+R+\Phi-\beta X^+-\one\lambda_2^T).$$
\end{frame}
\newtheorem{rem}{注}
\begin{frame}{子问题的求解--$Z$子问题 (续)}
问题\beq{Z subproblem}的等价形式:
\begin{equation*}
	\begin{array}{rl}
		\min\limits_Z & \Vert Z-\widetilde Z\Vert_F^2\\
		\st & Z^T\one=\rho,\quad Z\ge0.
	\end{array}
\end{equation*}\pause
分块
$$Z=[z_1,\ldots,z_n],\quad\widetilde Z=[\tilde z_1,\ldots,\tilde z_n].$$
\begin{equation}
	\Leftrightarrow\begin{array}{rl}
		\min\limits_{z_1,\ldots,z_n} & \sum\limits_{j=1}^n\Vert z_j-\tilde z_j\Vert^2\\
		\st & \one^Tz_j=\rho_j,\quad z_j\ge0,\quad j=1,\ldots,n.
	\end{array}
	\label{Z subproblem equal 2}
\end{equation}
\texttt{quadprog()}$\Rightarrow Z^+$.\pause
\begin{rem}
若未恰当引入分裂变量$Z$, $X$子问题的求解难度非常之大.
\end{rem}
\end{frame}

\subsection{拉格朗日乘子的更新}
\begin{frame}{拉格朗日乘子的更新}
受\tnewroman ALM算法启发, 更新策略可选为
\begin{equation}
	\Phi^+=\Phi-\beta(X^+-Z^+).\label{multiplier update}
\end{equation}
若带松弛因子$\alpha>0$, 则
\begin{equation}
	\Phi^+=\Phi-\alpha\beta(X^+-Z^+).\label{multiplier update relax}
\end{equation}
\end{frame}
\subsection{停机准则与\tnewroman{KKT}违反度}
\begin{frame}{停机准则与\tnewroman{KKT}违反度}
对$X$子问题:
$$X^{k+1}=\arg\min\limits_{X:X\one=\rho,\trace(X)=0}\mcL_A(X,Z^k,\Phi^k)$$
	\begin{itemize}
	\item 使用更新策略\beq{multiplier update}:
	$$\boxed{\color{red}{(Z^{k+1}-Z^k)(\beta I-R)}};$$
	\item 使用更新策略\beq{multiplier update relax}:
	$$\boxed{{\color{red}{(Z^{k+1}-Z^k)(\beta I-R)}}},\quad\boxed{{\color{red}{(1-\alpha)\beta(X^{k+1}-Z^{k+1})}}}.$$
	\end{itemize}
\end{frame}

\begin{frame}{停机准则与\tnewroman{KKT}违反度 (续)}
对$Z$子问题:
$$Z^{k+1}=\arg\min\limits_{Z:Z^T\one=\rho,Z\ge0}\mcL_A(X^{k+1},Z,\Phi^k)$$
\begin{itemize}
	\item 使用更新策略\beq{multiplier update}:
		$$\makebox{恰好此部分\tnewroman KKT违反度为0.}$$
	\item 使用更新策略\beq{multiplier update relax}:
		$$\boxed{{\color{red}{(1-\alpha)\beta(X^{k+1}-Z^{k+1})}}}.$$
	\end{itemize}
\end{frame}

\begin{frame}{停机准则与\tnewroman{KKT}违反度 (续)}
	\tnewroman KKT违反度:
	\begin{equation}\begin{array}{ll}
		t^{k+1}\triangleq\Vert X^{k+1}-Z^{k+1}\Vert_{\infty} & \makebox{\textbf{原始残差}},\\
		s^{k+1}\triangleq\Vert (Z^{k+1}-Z^k)(\beta I-R)\Vert_{\infty} & \makebox{\textbf{对偶残差}}.
	\end{array}\label{residual}\end{equation}\pause
	停机准则:
	\begin{enumerate}[情形1]
	\item $t^{k+1},s^{k+1}$都足够小; 
	\item 对某个$p^{k+1}\in(0,1)$, $p^{k+1}s^{k+1}+\left(1-p^{k+1}\right)t^{k+1}$足够小. 
	\end{enumerate}\pause\vspace{1em}
	我们使用\begin{equation}E^{k+1}=(1-p^{k+1})t^{k+1}+p^{k+1}s^{k+1}\label{KKT Violation}\end{equation}作为\textbf{\tnewroman{KKT}违反度}.
\end{frame}

\subsection{完整算法}
\begin{frame}{完整算法}
\begin{algorithm}[H]
	\floatname{algorithm}{算法}
	\caption{求解问题(4)的\tnewroman{ADMM}算法}
	\label{ADMM}
	\begin{algorithmic}[1]
		\REQUIRE $X^0,Z^0,\Phi^0$, $\beta^0$, $\epsilon$, $k:=0$, $s^0:=1,t^0:=1$, $\alpha>0 (\makebox{默认值为}1)$, $p^0\in(0,1)$.
		\ENSURE $X^k,Z^k,\Phi^k$
		\STATE $E^k=(1-p^{k})t^k+p^ks^k$;
		\WHILE{$E^k>\epsilon$}
		\STATE {由公式\beq{X multiplier},\beq{pre solve X}计算$X^{k+1}$;}
		\STATE {使用\tnewroman{MATLAB}内置函数\texttt{quadprog()}求解列子问题\beq{Z subproblem equal 2}得到$Z^{k+1}$;}
		\STATE {$\Phi^{k+1}=\Phi^k-\alpha\beta^k(X^{k+1}-Z^{k+1})$;}
		\STATE{由公式\beq{residual}计算$t^{k+1},s^{k+1}$;}
		\STATE{更新$\beta^k$得到$\beta^{k+1}$;}
		\STATE{更新$p^k$得到$p^{k+1}$;}
		\STATE{由公式\beq{KKT Violation}得到$E^{k+1}$;}
		\STATE{$k:= k+1$;}
	\ENDWHILE
	\end{algorithmic}
\end{algorithm}
\end{frame}


\section{收敛性分析}
\newtheorem{thm}{定理}
\begin{frame}{收敛性分析}
	\begin{thm}[充分性定理]
		假设算法\ref{ADMM}每一步$,Z$子问题均精确求解, 且产生的迭代序列$\{X^k\},\{Z^k\},\{\Phi^k\}$分别收敛到$X^*,Z^*,\Phi^*$, 满足$X^*=Z^*$. 则$(X^*,Z^*,\Phi^*)$为问题\beq{original problem matrix form}的稳定点.
	\end{thm}\pause
	\textbf{关键点}:
	\begin{enumerate}
		\item $X^*\in\arg\min\limits_{X\one=\rho,\trace(X)=0}f(X,Z^*)-\langle\Phi^*,X-Z^*\rangle.$\label{convergence1}
		\item $Z^*\in\arg\min\limits_{Z^T\one=\rho,Z\ge0}f(X^*,Z)-\langle\Phi^*,X^*-Z\rangle.$\label{convergence2}
	\end{enumerate}
\end{frame}

\section{数值实验}
\begin{frame}{数值实验}
使用\tnewroman{MATLAB}内置函数\texttt{randn()}和\texttt{abs()}生成随机问题. 小型问题以$n=3,4,5$各一问题为代表, 大型问题以$n=20,30,40$各一问题为代表. \\[0.5em]
小型问题中的常量取法:
$$\alpha=1,\quad \beta^k\equiv10^3,\quad \epsilon=10^{-8},\quad p^k\equiv0.5.$$
大型问题中的常量取法:
$$\alpha=1,\quad \beta^k\equiv10^4,\quad \epsilon=10^{-6},\quad p^k\equiv0.5.$$
\end{frame}
\renewcommand{\figurename}{图}
\renewcommand{\tablename}{表}
\setbeamertemplate{caption}[numbered]
\subsection{收敛曲线}
\begin{frame}{数值实验--收敛曲线}
\begin{figure}[H]
	\centering
			\includegraphics[width=.31\linewidth]{n4primalres.jpg}
				\includegraphics[width=.31\linewidth]{n4dualres.jpg}
					\includegraphics[width=.31\linewidth]{n4obj.jpg}
\end{figure}
\begin{center}
	\small\textcolor[rgb]{0,0.25,0.604}{图 1: }$n=4$收敛曲线 \\\footnotesize 左: 原始残差; 中: 对偶残差; 右: 目标函数值.
\end{center}
\end{frame}

\subsection{松弛因子$\alpha$}
\begin{frame}{数值实验--松弛因子$\alpha$}
$\alpha=0.1,0.2,\ldots,1.0. $
\only<1>{
\tnewroman
\begin{figure}[H]
		\centering
				\includegraphics[width=.31\linewidth]{alpha_iter3.jpg}
					\includegraphics[width=.31\linewidth]{alpha_time3.jpg}
						\includegraphics[width=.31\linewidth]{alpha_target3.jpg}
\end{figure}
\begin{center}
	\small\textcolor[rgb]{0,0.25,0.604}{图 2: }$n=3$, 不同的松弛因子\\\footnotesize 左: 迭代数; 中: 所耗时间; 右: 目标值.
\end{center}
}
\only<2>{\begin{figure}[H]
		\centering
				\includegraphics[width=.31\linewidth]{alpha_iter30.jpg}
					\includegraphics[width=.31\linewidth]{alpha_time30.jpg}
						\includegraphics[width=.31\linewidth]{alpha_target30.jpg}
\end{figure}
\begin{center}
	\small\textcolor[rgb]{0,0.25,0.604}{图 3: }$n=30$, 不同的松弛因子\\\footnotesize 左: 迭代数; 中: 所耗时间; 右: 目标值.
\end{center}}
\end{frame}

%\begin{frame}{数值实验--松弛因子$\alpha$}
%$\alpha=0.1,0.2,\ldots,1.0$.
%\begin{itemize}
%\item 不同的松弛因子可能会影响目标值.
%\item 松弛因子对小型问题不会有多大影响, 但却十分影响大型问题.
%\end{itemize}
%\end{frame}

\subsection{惩罚因子$\beta$}
\begin{frame}{数值实验--惩罚因子$\beta$}
\only<1>{\begin{center}
	\small\textcolor[rgb]{0,0.25,0.604}{表 1: }不同的惩罚因子
\end{center}
\tnewroman
\renewcommand\arraystretch{0.4}
\begin{table}[htbp]
    \centering
    \label{n3beta}
    \begin{tabular}{c|c|c|c|c}
        \hline
        \multirow{2}{*}{$\beta$} & \multicolumn{4}{c}{$n=3$}\\\cline{2-5}
          & 迭代数 & 所耗时间 (s) & KKT违反度 & 目标值\\\hline
        100 & \textbf{224} & \textbf{0.0043} & 5.46$\times10^{-9}$ & \textbf{1.1722}\\\hline
        1000 & 560 & 0.0095 & 8.02$\times10^{-9}$ & \textbf{1.1722}\\\hline
        10000 & 3998 & 0.0647 & \textbf{4.92$\mathbf{\times10^{-9}}$} & \textbf{1.1722}\\\hline
        100000 & 38377 & 0.5818 & 7.68$\times10^{-9}$ & \textbf{1.1722}\\\hline
        \hline
        \multirow{2}{*}{$\beta$} & \multicolumn{4}{c}{$n=5$}\\\cline{2-5}
          & 迭代数 & 所耗时间 (s) & KKT违反度 & 目标值\\\hline
        100 & 969 & 0.0465 & 5.00$\times10^{-9}$ & \textbf{2.7741}\\\hline
        1000 & \textbf{860} & \textbf{0.0384} & 9.54$\times10^{-9}$ & \textbf{2.7741}\\\hline
        10000 & 3857 & 0.1698 & \textbf{4.64$\mathbf{\times10^{-9}}$} & \textbf{2.7741}\\\hline
        100000 & 33470 & 1.4344 & 9.47$\times10^{-9}$ & \textbf{2.7741}\\\hline
        \hline
        \multirow{2}{*}{$\beta$} & \multicolumn{4}{c}{$n=30$}\\\cline{2-5}
          & 迭代数 & 所耗时间 (s) & KKT违反度 & 目标值\\\hline
        10000 & \textbf{207788} & \textbf{133.6154} & \textbf{6.00$\mathbf{\times10^{-7}}$} & \textbf{24.9461}\\\hline
        100000 & 2053406 & 1327.5987 & 7.08$\times10^{-7}$ & 24.9542\\\hline        
     \end{tabular}
\end{table}}
\onslide<2>{
\begin{figure}[htbp]
	\centering
	\includegraphics[width=.37\paperwidth]{primalres.jpg}
	\includegraphics[width=.37\paperwidth]{dualres.jpg}
	\label{loop}
\end{figure}
\begin{center}
	\small\textcolor[rgb]{0,0.25,0.604}{图 4: }$n=30,\beta=10^3$残差陷入循环\\\footnotesize 左: 原始残差; 右: 对偶残差.
\end{center}
}
\end{frame}

%\begin{frame}{数值实验--惩罚因子$\beta$}
%\begin{itemize}
%\item 惩罚因子需适当选取. 过大减缓算法收敛, 过小可能使算法失效.
%\item 适宜惩罚因子的值域问题维数正相关, 即问题维数越大, 应选取更大的惩罚因子.
%\end{itemize}
%\end{frame}

\subsection{测试问题}
\begin{frame}{数值实验--测试问题}
给定一数对$(p,q):p\ne q,p,q\in\{1,2,\ldots,n\}$, 定义$R$为
\begin{equation}r_{ij}=\left\{\begin{array}{ll}
	1, & i=p,j=q\makebox{或}i=q,j=p,\\
	0, & \makebox{其它}.
\end{array}\right.\end{equation}
$\rho:=\one$. 
\begin{equation}\begin{array}{rl}
	\min\limits_X & 2x_{pq}+2\sum\limits_ix_{ip}x_{iq}\\
	\st & X\one=\rho,X^T\one=\rho,X\ge0,\trace(X)=0.
\end{array}\end{equation}
最优值为\tnewroman0.
	\begin{center}
		\small\textcolor[rgb]{0,0.25,0.604}{表 2: }测试问题, $n=5,10,15,20$
	\end{center}
	\tnewroman
	\begin{table}[H]
		\centering
		\label{test}
		\begin{tabular}{c|c|c|c|c}
			\hline
			$n$ & 迭代数 & 所耗时间 (s) & KKT违反度 & 目标值\\\hline
			5 & 3187 & 0.1536 & 3.00$\times10^{-9}$ & -1.44$\times10^{-11}$\\\hline
			10 & 1447 & 0.1766 & 1.65$\times10^{-9}$ &  -4.94$\times10^{-15}$\\\hline
			15 & 2243 & 0.3284 & 2.30$\times10^{-9}$ & 6.54$\times10^{-13}$\\\hline
			20 & 2030 & 0.3299 & 4.53$\times10^{-9}$ & -8.07$\times10^{-13}$\\\hline
		\end{tabular}
	\end{table}
\end{frame}

\subsection{与求解非凸二次规划的算法比较}
\begin{frame}{数值实验--与求解非凸二次规划的算法比较}
\tnewroman{MATLAB}内置函数\texttt{fmincon()}, 调用算法\texttt{'sqp'}求解问题\beq{original problem matrix form 2}和\beq{original problem matrix form 1}, 设置停机准则\texttt{ConstraintTolerance}为上文默认值.

\begin{figure}[H]
	\centering
			\subfigure[所耗时间]{
				\begin{minipage}[b]{.45\linewidth}
					\includegraphics[width=\linewidth]{already_time.jpg}
				\end{minipage}}
			\subfigure[目标值]{
						\begin{minipage}[b]{.45\linewidth}
							\includegraphics[width=\linewidth]{already_target.jpg}
						\end{minipage}}
\end{figure}
\begin{center}
		\small\textcolor[rgb]{0,0.25,0.604}{图 5: }算法\ref{ADMM}和\tnewroman SQP运行时间对比
	\end{center}
\textbf{结论}: 对于大型问题, 算法\ref{ADMM}更具优势. 算法\ref{ADMM}的缺点在于, 我们需要根据问题的维度不断重新设置惩罚因子$\beta$.
\end{frame}


\section{总结}
\begin{frame}{总结}
\begin{itemize}
\item 针对问题\beq{original problem matrix form}特殊结构设计了\tnewroman{ADMM}算法.
\item 证明了一定条件下\tnewroman{ADMM}算法\ref{ADMM}收敛到问题\beq{original problem matrix form}的稳定点.
\item 在随机生成的问题上进行数值实验; 讨论了多个人工给定因子对算法效果的影响; 与求解非凸二次规划的算法做了比较, 得出算法\ref{ADMM}在大型问题上更具优势的结论.
\end{itemize}
\textbf{特别地}:
\begin{itemize}
\item 引入变量分裂约束和目标函数, 使问题构造方便算法设计.
\item 使用\tnewroman ADMM算法求解带特殊约束的问题. 此类约束在最优运输问题中很常见.
\item 将$Z$子问题划分成若干个互不相关的小子问题求解.
\item 给出了算法收敛到稳定点的充分性定理.
\end{itemize}
\end{frame}
\logo{}
\section*{致谢}
\begin{frame}
\begin{center}
	\Large 感谢聆听!\\[0.5em]
	\small\url{huyukuan2015@tongji.edu.cn}
\end{center}
\end{frame}
\end{document}
